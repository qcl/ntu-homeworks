\documentclass[12pt]{article}
\usepackage{MinionPro}
\usepackage{CJK}
%\usepackage[scaled=0.85]{beramono}  %%% scaled=0.775
\usepackage[T1]{fontenc}
\usepackage{graphicx,booktabs,tabularx,psfrag}
\usepackage{xcolor}
\usepackage[a4paper]{geometry}
\usepackage{url}

\usepackage{pstool}

\usepackage{graphicx}
\begin{document}
\begin{CJK}{UTF8}{cwmb}
\renewcommand{\figurename}{圖}

\voffset=-1cm
\textwidth=5.6in
\textheight=9.2in

\newenvironment{num}
 {\leftmargini=6mm\leftmarginii=8mm
  \begin{enumerate}\itemsep=-2pt}
 {\end{enumerate}}

\newenvironment{sol}
 {\begin{quote}\mbox{}\llap{\color{blue}{解答:}\rule{10mm}{0pt}}\hspace*{-4pt}}{\end{quote}}


\thispagestyle{empty}
\fontsize{12}{20pt}\selectfont
\begin{center}
{\large\CJKfamily{cwyb}{經濟學原理下, 習題8}}\\[3mm]
劉彥佑 (R99628130)\\
李卿澄 (B97501046)\\
黃博億 (B99101014)\\
王祉婷 (B00704056)
\end{center}

\begin{num}
\item 
	\begin{num}
		\item 重貼現率上升,代表銀行向央行借錢的利率上升,會降低銀行借錢的意願,理論準備貨幣會相對減少,但一般銀行不會向央行借錢,所以不會直接影響到準備貨幣。
		\item 若重貼現率上升導致準備貨幣相對減少,則M2貨幣供給也會相對下降。
		\item 央行出售定期存單,猶如央行把市場上的現金收回,將使準備貨幣減少、市場資金減少。市場資金減少會使利率上升。
	\end{num}
\item 重貼現率調升,會使貨幣供給相對減少,而長期來看,貨幣供給量決定了物價的水準,因此調升重貼現率將有助於抑制物價上漲。
\item 
	\begin{num}
		\item 課本214頁中提到,臺灣在1980-2005年間的平物價膨脹率是3.07\%,而由圖來看,物價膨脹率與貨幣供給成長率幾乎完全正相關,因此臺灣的貨幣供給成長率較接近A國。
		\item 由貨幣供需均衡條件$\pi=\frac{\Delta M^s}{M^s}-\frac{\Delta m}{m}$,可以推得$\frac{\Delta m}{m}=\frac{\Delta M^s}{M}-\pi$。
		\item $m=aY$,由a為常數可知$\frac{\Delta Y}{Y} = \frac{\Delta m}{m} = 4\%$,且$\pi = 2\%$,貨幣供給成長率$\frac{\Delta M^s}{M^s} = \pi + \frac{\Delta m}{m} = 2\% + 4\% = 6\%$。
	\end{num}
\end{num}
\end{CJK}
\end{document}
