\documentclass[12pt]{article}
\usepackage{MinionPro}
\usepackage{CJK}
%\usepackage[scaled=0.85]{beramono}  %%% scaled=0.775
\usepackage[T1]{fontenc}
\usepackage{graphicx,booktabs,tabularx,psfrag}
\usepackage{xcolor}
\usepackage[a4paper]{geometry}
\usepackage{url}

\usepackage{pstool}

\usepackage{graphicx}
\begin{document}
\begin{CJK}{UTF8}{cwmb}
\renewcommand{\figurename}{圖}

\voffset=-1cm
\textwidth=5.6in
\textheight=9.2in

\newenvironment{num}
 {\leftmargini=6mm\leftmarginii=8mm
  \begin{enumerate}\itemsep=-2pt}
 {\end{enumerate}}

\newenvironment{sol}
 {\begin{quote}\mbox{}\llap{\color{blue}{解答:}\rule{10mm}{0pt}}\hspace*{-4pt}}{\end{quote}}


\thispagestyle{empty}
\fontsize{12}{20pt}\selectfont
\begin{center}
{\large\CJKfamily{cwyb}{經濟學原理下, 習題3}}\\[3mm]
劉彥佑 (R99628130)\\
李卿澄 (B97501046)\\
黃博億 (B99101014)\\
王祉婷 (B00704056)
\end{center}

\begin{num}
\item $10,000(1+R)^5 = 10,000(1+0.04)^5 \cong 12,167$元。
\item $\frac{4,000,000}{(1+R)^3} = \frac{4,000,000}{(1+0.05)^3} \cong 3,455,350$元。
\item 
	\begin{num}
		\item 名目儲蓄為$(b_1+m_1)-(b_0+m_0) = (50+2)-(60+1) = -9$ 萬元。
		\item 實質儲蓄為$\frac{b_1+m_1}{p_1} - \frac{b_0+m_0}{p_0}$, 設$p_0$為1萬元,則可以推得實質儲蓄為$\frac{b_1+m_1}{p_0(1+\pi)} - \frac{b_0+m_0}{p_0} = \frac{50+2}{1.1} - (60+1) \cong -13.73$單位$p_0$。
	\end{num}
\item
	\begin{num}
		\item $b_0(1+R)+m_0+p_1y_1 = p_1c_1+b_1+m_1 \rightarrow 60(1+0.05) + 1 + 50 = 45 + b_1 + 2$ 並可求得$b_1=67$萬元。 
		\item $(b_1+m_1)-(b_0+m_0)=(67+2)-(60+1)=8$萬元。
		\item 實質儲蓄為$\frac{b_1+m_1}{p_1} - \frac{b_0+m_0}{p_0}$, 設$p_0$為1萬元,則可以推得實質儲蓄為$\frac{b_1+m_1}{p_0(1+\pi)} - \frac{b_0+m_0}{p_0} = \frac{67+2}{1.02} - (60+1) \cong 6.6471$單位$p_0$。
		\item $b_0(1+R)+m_0+p_1y_1+\frac{p_2y_2}{(1+R)} = 60(1+0.05) + 1 + 50 + \frac{55}{1+0.05} \cong 166.3810$萬元。
		\item 由(a)(b)已知今年結束時$m_1=2,b_1=67$,並配合(d)之資訊可以算出明年可支配之預算為$67(1+0.05)+2+55=127.35$萬元。
	\end{num}
\item $100(1+R)^{2010-1980}=100(1+0.05)^{30}=432.19$元。
\item 儲蓄為所得減去消費支出,其儲蓄為$100-40=60$萬元。
\end{num}

\end{CJK}
\end{document}
