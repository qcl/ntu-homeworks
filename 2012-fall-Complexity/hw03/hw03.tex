\documentclass[11pt]{article}
\usepackage{amsmath}
\usepackage{CJK}
\title{\textbf{Theory of Computation}}
\author{Homework 3\\
					\\
		Qing-Cheng Li\\
		R01922024}
\date{\today}
\usepackage{graphicx}
\begin{document}
\maketitle
\section{Problem 1}
We can write a nondeterministic polynomial-time algorithm which takes a $D-SAT$ instance and 2 proposed truth assignments as input, if the $D-SAT$ instance evelates 2 assignments to true, the algorithm outputs $yes$; otherwise outputs $no$. This runs in polynomial time, so $D-SAT$ is in \emph{NP}. We reduce $SAT$ to $D-SAT$ as follows. Let $\phi$ donates an instance of $SAT$, we add a new variable $y$, convert $\phi$ to $\phi' = \phi \wedge (y \vee \neg y)$, a $D-SAT$ instance. If $\phi$ has at least 1 statisfying assignment, then $\phi \wedge (y \vee \neg y)$ is true for $y=1$ or $y=0$, so $\phi'$ has at least 2 statisfying assignments. If $\phi$ has no statisfying assignment, then $\phi'$ is also has no statisfying assignment. So $D-SAT$ is \emph{NP-Complete}.

\section{Problem 2}
We can write a nondeterministic polynomial-time algorithm which takes a undirected graph $G$ and nodes $n_s$, $n_e$ as input, nondeterministically choose path from $n_s$ to $n_e$, if this path is a Hamiltonian path, this algorithm outputs $yes$, otherwise outputs $no$. This runs in polynomial time, so $SE-Hamiltonian Path$ is in \emph{NP}. We reduce $Hamiltonian Cycle$ to $SE-Hamiltonian Path$ as follows. Given a undirected graph $G$, we can choose a node $n$ from $G$, let this node $n$ be $n_s$ and $n_e$ in $SE-Hamiltonian Path$. If this graph $G$ has a Hamiltonian Cycle, then there is a Hamiltonian Path from $n$ to $n$ on $G$. If $G$ has no Hamiltonian Path from $n$ to $n$, $G$ has no Hamiltonian Cycle. So $SE-Hamiltonian Path$ is \emph{NP-Complete}. 

\end{document}
