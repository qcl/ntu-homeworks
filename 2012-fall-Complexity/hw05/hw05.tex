\documentclass[11pt]{article}
\usepackage{amsmath}
\usepackage{CJK}
\title{\textbf{Theory of Computation}}
\author{Homework 5\\
					\\
		Qing-Cheng Li\\
		R01922024}
\date{\today}
\usepackage{graphicx}
\begin{document}
\maketitle
\section{}
Let $t$ be any negative real number.
\begin{align*}
Pr[X \leq (1-\theta)pn] &= Pr[e^{tX} \geq e^{t(1-\theta)pn}] \leq \frac{E[e^{tX}]}{e^{t(1-\theta)pn}}
\end{align*}
(Markov's inequality, with $k = e^{t(1-\theta)pn}/E[e^{tX}]$)
\begin{align*}
Pr[X \leq (1-\theta)pn] & \leq \frac{E[e^{tX}]}{e^{t(1-\theta)pn}} \leq \frac{e^{(e^t-1)pn}}{e^{t(1-\theta)pn}}
\end{align*}
Let $t = ln(1-\theta) < 0$
\begin{align*}
Pr[X \leq (1-\theta)pn] & \leq \frac{e^{(e^t-1)pn}}{e^{t(1-\theta)pn}} = (\frac{e^{-\theta}}{(1-\theta)^{(1-\theta)}})^{pn}
\end{align*}
\begin{align*}
(1-\theta)ln(1-\theta) &= (1-\theta)(-\theta-\frac{\theta^2}{2}-\frac{\theta^3}{3}-...)\\
&= -\theta+\frac{\theta^2}{2}+(\frac{1}{2}-\frac{1}{3})\theta^3+(\frac{1}{3}-\frac{1}{4})\theta^4... \geq -\theta+\frac{\theta^2}{2}\\
e^{(1-\theta)ln(1-\theta)} &= (1-\theta)^{(1-\theta)} \geq e^{-\theta+\frac{\theta^2}{2}}
\end{align*}
\begin{align*}
Pr[X \leq (1-\theta)pn] & \leq (\frac{e^{-\theta}}{(1-\theta)^{(1-\theta)}})^{pn} \leq (\frac{e^{-\theta}}{e^{-\theta+\frac{\theta^2}{2}}})^{pn} = e^{\frac{-\theta^2pn}{2}}
\end{align*}

\section{}
If $L$ is in \emph{BPP}, then there is a probabilistic polynomial-time algorithm $A$ for $L$ running in polynomial-time $p(n)$. If we want to know whether a input $x \in \{0,1\}^n$ is in $L$ or not, we need to calculate the probability below:  
\begin{align*}
Pr[A(x;r) accepts]
\end{align*}
$r \in \{0,1\}^{m(n)}$ is nondeterministic choices of computation path. $A(x;r)$ is running algorithm $A$ on input $x$ with computation path $r$. 
\begin{align*}
Pr[A(x;r) accepts] &= \frac{\sum_{r \in \{0,1\}^{m(n)}}A(x;r)}{2^{m(n)}}
\end{align*}
We can compute it in deterministic time $2^{m(n)} \times p(n)$, so $BPP \subseteq EXP $.
\end{document}
